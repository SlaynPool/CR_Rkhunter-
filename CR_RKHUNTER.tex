
\documentclass[10pt,a4paper]{article}
\usepackage[utf8]{inputenc}
\usepackage[french]{babel}
\usepackage[left=2cm,right=2cm,top=2cm,bottom=2cm]{geometry}
\usepackage{hyperref}
\usepackage{graphicx}
% \usepackage{fancyhdr}
% \fancyhead{}
% \fancyfoot{}
% \fancyhead[L]{\includegraphics[scale=0.03]{../image/logo_iutbeziers.png}}
% \fancyhead[C]{Rapport de Stage - SuperBeeLive}
% 
% \fancyfoot[L]{\small Olivia SERENELLI-PESIN \normalsize}
% \fancyfoot[R]{\thepage/\pageref{LastPage}}
% %\fancyfoot[C]{\includegraphics[scale=0.03]{../images/logo/logo_abeille.png}}
% \renewcommand{\footrulewidth}{0pt}
% \renewcommand{\headrulewidth}{0,4pt}
% 
% %Indique qu'il faut appliquer le style sur tout le doc 
% \makeatletter
% \let\ps@plain=\ps@fancy
% \makeatother

%opening
\title{CR RKHUNTER}
\author{Nicolas Vadkerti}
\usepackage{listingsutf8} % Required for inserting code snippets
\usepackage[usenames,dvipsnames]{color} % Required for specifying custom colors and referring to colors by name

\definecolor{DarkGreen}{rgb}{0.0,0.4,0.0} % Comment color
\definecolor{highlight}{RGB}{255,251,204} % Code highlight color

\lstdefinestyle{Style1}{ % Define a style for your code snippet, multiple definitions can be made if, for example, you wish to insert multiple code snippets using different programming languages into one document
language=Perl, % Detects keywords, comments, strings, functions, etc for the language specified
backgroundcolor=\color{highlight}, % Set the background color for the snippet - useful for highlighting
basicstyle=\footnotesize\ttfamily, % The default font size and style of the code
breakatwhitespace=false, % If true, only allows line breaks at white space
breaklines=true, % Automatic line breaking (prevents code from protruding outside the box)
captionpos=b, % Sets the caption position: b for bottom; t for top
commentstyle=\usefont{T1}{pcr}{m}{sl}\color{DarkGreen}, % Style of comments within the code - dark green courier font
deletekeywords={}, % If you want to delete any keywords from the current language separate them by commas
%escapeinside={\%}, % This allows you to escape to LaTeX using the character in the bracket
firstnumber=1, % Line numbers begin at line 1
frame=single, % Frame around the code box, value can be: none, leftline, topline, bottomline, lines, single, shadowbox
frameround=tttt, % Rounds the corners of the frame for the top left, top right, bottom left and bottom right positions
keywordstyle=\color{Blue}\bf, % Functions are bold and blue
morekeywords={}, % Add any functions no included by default here separated by commas
numbers=left, % Location of line numbers, can take the values of: none, left, right
numbersep=10pt, % Distance of line numbers from the code box
numberstyle=\tiny\color{Gray}, % Style used for line numbers
rulecolor=\color{black}, % Frame border color
showstringspaces=false, % Don't put marks in string spaces
showtabs=false, % Display tabs in the code as lines
stepnumber=5, % The step distance between line numbers, i.e. how often will lines be numbered
stringstyle=\color{Purple}, % Strings are purple
tabsize=2
}

\newcommand{\insertcode}[2]{\begin{itemize}\item[]\lstinputlisting[caption=#2,label=#1,style=Style1]{#1}\end{itemize}} 


% \insertcode{"Scripts/example.pl"}{Nena would be proud.} 

\begin{document}

\maketitle


\url{https://github.com/SlaynPool/CR_Rkhunter-}


\section{Définitions}
\subsection{Rootkit}
Un rootKit est un, ou plusieurs logiciels dont le but est de péréniser un accés. Le plus souvent, il sert a dissimuler la porte ouverte sur une machine une fois l'accés à celle-ci obtenue. 
\subsection{0days}
Une faille 0 days est un bug d'un logiciel qui laisse des failles sur une machines. On l'appelle 0days car il n'est pas encore patché voir connu par les dévellopeurs du logiciel. Les 0days sont donc très dangereuse. De fait, quand une faille 0days est découverte, on ne peut savoir depuis quand elle est realisable, ni si quelqu'un l'a exploité.
On peut citer Hearthbleed qui utiliser une faille OpenSSL comme exemple d'ancienne faille 0 days qui à fais du bruit.


\subsection{Virus}
s
Le principe d'un virus Informatique est exactement le même q'un virus Biologique. On peut donc le définir comme un programe qui à pour but de se répliquer, souvent dans d'autres hôtes d'un réseau, dans le but de prendre le controle des machines. L'utilisations peut etre par exemple d'obtenir des bots pour des attaques par deni de service.




\subsection{CVE}

 Common Vulnerabilities and Exposures
 Le CVE est un dictionnaire de faille Informatique connu maintenue par le MITRE, un organisme publique Americain.
 Ce dictionnaire sert de ``norme '' aux outils de sécurté comme les antivirus par exemple.
 
 \subsection{La notion de Hash}
 Le principe d'un hash, est simple à comprendre. L'idée est d'avoir un algorithme qui prend un fichier quelconque en entrée, et renvoie une signature de celui de taille constante et surout unique.
 L'idée est donc de, ci je calcule le hash d'un fichier, si je modifie mon fichier, le hash sera completement different.
 On peut donc s'assurer que personne n'a modifier mon fichier grace à ce hash.
 Un algorithme de hashage n'est plus considéré comme fiable, à partir du moment où pour deux fichiers differents, nous somme capable de générer des hashs partiellement similaires ou entierement similaires
 
\section{Protection contre les Rootkit:  le cas de rkhunter}
\subsection{   Installation de rkhunter }
\insertcode{commande/installRkhunter.txt}{Installation de RKHUNTER}




\end{document}

